\documentclass[]{../unsthesis}
% \documentclass[nocolor]{../unsthesis}

\graphicspath{ {./res/} }

\renewcommand{\figurecaptionname}{My Figure}
\renewcommand{\listingcaptionname}{My Listing}
\renewcommand{\tablecaptionname}{My Table}
\renewcommand{\theoremcaptionname}{My Theorem}
\renewcommand{\contentscaptionname}{My Contents}
\renewcommand{\literaturecaptionname}{My Literature}

\renewcommand{\ThesisAuthor}{Johnny Dough}
\renewcommand{\ThesisTitle}{Indisputable proof of the Riemann hypothesis}
\renewcommand{\ThesisType}{Master's thesis}
\renewcommand{\DegreeType}{Postgraduate academic studies}
\renewcommand{\UniversityName}{Insert university name}
\renewcommand{\FacultyName}{Insert name of college or department}
\renewcommand{\Date}{2025}
\renewcommand{\City}{My City}
\renewcommand{\UniversityLogoPath}{logouni.png}
\renewcommand{\FacultyLogoPath}{logofac.png}

\begin{document}

\maketitlepage

% Each section is in its own .tex file to improve readability.
% \cleardoublepage is used to add a blank page so that the proceeding
% chapter starts on the odd page (i.e. on the "left" page). Doing this
% makes sense only if you're doing `twoside` documents, which you should
% because it saves paper. 

\cleardoublepage
\pagestyle{empty}

{
    \section*{\MakeUppercase{Кључна документацијска информација}}
    \small
    \begin{longtable}{| p{.35\textwidth} | p{.65\textwidth} |}
        \hline
        Редни број, \textbf{РБР}:                       & \\ \hline
        Идентификациони број, \textbf{ИБР}:             & \\ \hline
        Тип документације, \textbf{ТД}:                 & монографска публикација \\ \hline
        Тип записа, \textbf{ТЗ}:                        & текстуални штампани документ \\ \hline
        Врста рада, \textbf{ВР}:                        & дипломски рад \\ \hline
        Аутор, \textbf{АУ}:                             & Име Презиме \\ \hline
        Ментор, \textbf{МН}:                            & Име и титула ментора \\ \hline
        Наслов рада, \textbf{НР}:                       & Наслов рада \\ \hline
        Језик публикације, \textbf{ЈП}:                 & српски \\ \hline
        Језик извода, \textbf{ЈИ}:                      & српски / енглески \\ \hline
        Земља публиковања, \textbf{ЗП}:                 & Србија \\ \hline
        Уже географско подручје, \textbf{УГП}:          & Назив региона нпр. Војводина \\ \hline
        Година, \textbf{ГО}:                            & 2025 \\ \hline
        Издавач, \textbf{ИЗ}:                           & ауторски репринт \\ \hline
        Место и адреса, \textbf{МА}:                    & град, факултет, адреса \\ \hline
        Физички опис рада, \textbf{ФО}:                 & $n$ поглавља, $k$ страна, $m$ цитата, $p$ табела, $l$ слика, $r$ исечака \\ \hline
        Научна област, \textbf{НО}:                     & Софтверско инжењерство и информационе технологије  \\ \hline
        Научна дисциплина, \textbf{НД}:                 & Софтверско инжењерство \\ \hline
        Предметна одредница / кључне речи, \textbf{ПО}: & кључна реч 1, кључна реч 2, кључна реч 3 \\ \hline
        \textbf{УДК}                                    & \\ \hline
        Чува се, \textbf{ЧУ}:                           & нпр. Библиотека вашег факултета, адреса, град \\ \hline
        Важна напомена, \textbf{ВН}:                    & \\ \hline
        Извод, \textbf{ИЗ}:                             & Lorem ipsum dolor sit amet, consectetur adipiscing elit, sed do eiusmod tempor incididunt ut labore et dolore magna aliqua. Ut enim ad minim veniam, quis nostrud exercitation ullamco laboris nisi ut aliquip ex ea commodo consequat. Duis aute irure dolor in reprehenderit in voluptate velit esse cillum dolore eu fugiat nulla pariatur. Excepteur sint occaecat cupidatat non proident, sunt in culpa qui officia deserunt mollit anim id est laborum.    \\ \hline
        Датум прихватања теме, \textbf{ДП}:             & \\ \hline
        Датум одбране, \textbf{ДО}:                     & \\ \hline
        Чланови комисије, \textbf{КО}:                  & \\ \hline
        \quad председник                                & Име и титула председника комисије \\ \hline 
        \quad члан                                      & Име и титула члана комисије \\ \hline 
        \quad ментор                                    & Име и титула ментора \\ \hline 
        \multicolumn{2}{|l|}{Потпис ментора} \\[25pt] \hline
    \end{longtable}
}

{
    \newpage
    \section*{\MakeUppercase{Key words documentation}}
    \small
    \begin{longtable}{| p{.35\textwidth} | p{.65\textwidth} |}
        \hline
        Accession number, \textbf{ANO}:                 & \\ \hline
        Identification number, \textbf{INO}:            & \\ \hline
        Document type, \textbf{DT}:                     & monographic publication \\ \hline
        Type of record, \textbf{TR}:                    & textual material \\ \hline
        Contents code, \textbf{CC}:                     & bachelor thesis \\ \hline
        Author, \textbf{AU}:                            & Name Surname \\ \hline
        Mentor, \textbf{MN}:                            & Mentor's name and title \\ \hline
        Title, \textbf{TI}:                             & Title \\ \hline
        Language of text, \textbf{LT}:                  & Main Language / Second Language (if any) \\ \hline
        Language of abstract, \textbf{LA}:              & Main Language / Second Language (if any) \\ \hline
        Country of publication, \textbf{CP}:            & Your Country \\ \hline
        Locality of publication, \textbf{LP}:           & Your Region \\ \hline
        Publication year, \textbf{PY}:                  & 2025 \\ \hline
        Publisher, \textbf{PB}:                         & author's reprint \\ \hline
        Publication place, \textbf{PP}:                 & Your College's Address \\ \hline
        Physical description, \textbf{PD}:              & $n$ chapters, $k$ pages, $m$ citations, $p$ tables, $l$ pictures, $r$ listings \\ \hline
        Scientific field, \textbf{SF}:                  & Software Engineering and Information Technologies  \\ \hline
        Scientific discipline, \textbf{SD}:             & Software Engineering \\ \hline
        Subject / Keywords, \textbf{S/KW}:              & keyword1, keyword2, keyword3 \\ \hline
        \textbf{UDC}                                    & \\ \hline
        Holding data, \textbf{HD}:                      & Your college's library address, for example \\ \hline
        Note, \textbf{N}:                               & \\ \hline
        Abstract, \textbf{AB}:                          & Lorem ipsum dolor sit amet, consectetur adipiscing elit, sed do eiusmod tempor incididunt ut labore et dolore magna aliqua. Ut enim ad minim veniam, quis nostrud exercitation ullamco laboris nisi ut aliquip ex ea commodo consequat. Duis aute irure dolor in reprehenderit in voluptate velit esse cillum dolore eu fugiat nulla pariatur. Excepteur sint occaecat cupidatat non proident, sunt in culpa qui officia deserunt mollit anim id est laborum.    \\ \hline
        Accepted by sci. Board on, \textbf{ASB}:        & \\ \hline
        Defended on, \textbf{DE}:                       & \\ \hline
        Defense board, \textbf{DB}:                     & \\ \hline
        \quad president                                 & Defense board president's name and title \\ \hline 
        \quad member                                    & Defense board member's name and title \\ \hline 
        \quad mentor                                    & Mentor's name and title \\ \hline 
        \multicolumn{2}{|l|}{Mentor's signature} \\[25pt] \hline
    \end{longtable}
}

\newpage % Without this, the last page of this section may show its page number.
\pagestyle{plain} % Instead of `plain` you can do whichever you like, just keep it consistent.

\cleardoublepage
\tableofcontents

\cleardoublepage
\section{Introduction} \label{01_introduction}
Lorem ipsum dolor sit amet, consectetur adipiscing elit, sed do eiusmod tempor incididunt ut labore et dolore magna aliqua. Ut enim ad minim veniam, quis nostrud exercitation ullamco laboris nisi ut aliquip ex ea commodo consequat. Duis aute irure dolor in reprehenderit in voluptate velit esse cillum dolore eu fugiat nulla pariatur. Excepteur sint occaecat cupidatat non proident, sunt in culpa qui officia deserunt mollit anim id est laborum.
\cite{maggiore2012virtual}
\cite{baumgarte1983new}

\subsection{Sub-Introduction}
\textit{Lorem}\footnote{Lorem means lorem} ipsum dolor sit amet, consectetur adipiscing elit, sed do eiusmod tempor incididunt ut labore et dolore magna aliqua.
\\
Ut enim ad minim veniam, quis nostrud exercitation ullamco laboris nisi ut aliquip ex ea commodo consequat. Duis aute irure dolor in reprehenderit in voluptate velit esse cillum dolore eu fugiat nulla pariatur. Excepteur sint occaecat cupidatat non proident, sunt in culpa qui officia deserunt mollit anim id est laborum.

Lorem ipsum dolor sit amet, consectetur adipiscing elit, sed do eiusmod tempor incididunt ut labore et dolore magna aliqua. Ut enim ad minim veniam, quis nostrud exercitation ullamco laboris nisi ut aliquip ex ea commodo consequat. Duis aute irure dolor in reprehenderit in voluptate velit esse cillum dolore eu fugiat nulla pariatur. Excepteur sint occaecat cupidatat non proident, sunt in culpa qui officia deserunt mollit anim id est laborum.

\begin{theorem}
For every real number $x$, $x > x - 1$.
\end{theorem}

See also listing \ref{lst:print}. And also see section \ref{01_introduction}.

\begin{lstlisting}[language=C++, caption={Code listing. If you disable color, it will be all in black and white.}, label={lst:print}]
#include <asyncbufio.h>

void main() {
    AsyncBuffer* b = new AsyncBuffer(
        500 * sizeof(char), 
        NULL, 
        2
    );
    return 0;
}
\end{lstlisting}


\cleardoublepage
\section{Math theory} \label{02_math_theory}

Lorem ipsum dolor sit amet, consectetur adipiscing elit, sed do eiusmod tempor incididunt ut labore et dolore magna aliqua. Ut enim ad minim veniam, quis nostrud exercitation ullamco laboris nisi ut aliquip ex ea commodo consequat. Duis aute irure dolor in reprehenderit in voluptate velit esse cillum dolore eu fugiat nulla pariatur. Excepteur sint occaecat cupidatat non proident, sunt in culpa qui officia deserunt mollit anim id est laborum.
Lorem ipsum dolor sit amet, consectetur adipiscing elit, sed do eiusmod tempor incididunt ut labore et dolore magna aliqua. Ut enim ad minim veniam, quis nostrud exercitation ullamco laboris nisi ut aliquip ex ea commodo consequat. Duis aute irure dolor in reprehenderit in voluptate velit esse cillum dolore eu fugiat nulla pariatur. Excepteur sint occaecat cupidatat non proident, sunt in culpa qui officia deserunt mollit anim id est laborum.

Lorem ipsum dolor sit amet, consectetur adipiscing elit, sed do eiusmod tempor incididunt ut labore et dolore magna aliqua. Ut enim ad minim veniam, quis nostrud exercitation ullamco laboris nisi ut aliquip ex ea commodo consequat. Duis aute irure dolor in reprehenderit in voluptate velit esse cillum dolore eu fugiat nulla pariatur. Excepteur sint occaecat cupidatat non proident, sunt in culpa qui officia deserunt mollit anim id est laborum.
Lorem ipsum dolor sit amet, consectetur adipiscing elit, sed do eiusmod tempor incididunt ut labore et dolore magna aliqua. Ut enim ad minim veniam, quis nostrud exercitation ullamco laboris nisi ut aliquip ex ea commodo consequat. Duis aute irure dolor in reprehenderit in voluptate velit esse cillum dolore eu fugiat nulla pariatur. Excepteur sint occaecat cupidatat non proident, sunt in culpa qui officia deserunt mollit anim id est laborum.
Lorem ipsum dolor sit amet, consectetur adipiscing elit, sed do eiusmod tempor incididunt ut labore et dolore magna aliqua. Ut enim ad minim veniam, quis nostrud exercitation ullamco laboris nisi ut aliquip ex ea commodo consequat. Duis aute irure dolor in reprehenderit in voluptate velit esse cillum dolore eu fugiat nulla pariatur. Excepteur sint occaecat cupidatat non proident, sunt in culpa qui officia deserunt mollit anim id est laborum.

See also figure \ref{fig:satCirclePolygon}.
This is ,,an'' aligned equation:

\begin{align}
    \begin{split}
        \vec{v} = {} & \vec{x}
    \end{split}                                    \\
    \begin{split}
        \vec{a} ={} & \vec{v} = \vec{x}
    \end{split} \\
    \vec{a} ={} & \frac{\vec{F}}{m}
\end{align}

Here's a list:

\begin{itemize}
    \item Item $1$
    \item Item $1^2 + 1$
\end{itemize}

Here's a regular equation:

\begin{equation}
    y = ax + b
\end{equation}

\begin{figure}
    \centering
    \begin{overpic}[width=0.5\textwidth]{res/satCirclePolygon.pdf} % ,grid,tics=10
        \put (16,53) {$p_1$}
        \put (39,52) {$l$}
        \put (56,41) {$p_k$}
        \put (75,23) {$p_2$}
    \end{overpic}
    \caption{Vector graphics made in external tool (drawio), saved as .pdf.
    Labels over the image are defined in the tex document.
    That way, readers can copy the text and the text is vectorized. }
    \label{fig:satCirclePolygon}
\end{figure}

\cleardoublepage
\cleardoublepage
\bibliographystyle{unsrt}
\bibliography{refs}

\end{document}
